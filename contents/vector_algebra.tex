\section{线性代数}

\subsection{行列式}
\begin{exercise}
	已知数 $18055$, $83282$, $61042$, $48576$, $57776$ 都可被 $23$ 整除, 证明:行列式
	\[\begin{vmatrix}
	1 & 8 & 0 & 5 & 5 \\
	8 & 3 & 2 & 8 & 3 \\
	6 & 1 & 0 & 4 & 2 \\
	4 & 8 & 5 & 7 & 6 \\
	5 & 7 & 7 & 7 & 6 \\
	\end{vmatrix}\]
	也可被 $23$ 整除.
\end{exercise}

\begin{proofs}
	令 $\displaystyle C_5=\sum_{i=1}^{5}10^{5-i}\cdot C_i$, 则有 $\displaystyle 23\bigm| a_{5j},\quad (j=1,2,3,4,5)$.
\end{proofs}

\begin{exercise} \label{ex:vec:123x}
    计算
    \[
    \begin{vmatrix}
    1 & 2 & 3 & 4 \\
    x & 1 & 2 & 3 \\
    x & x & 1 & 2 \\
    x & x & x & 1
    \end{vmatrix}.
    \]
\end{exercise}

\begin{solution}
    \begin{align*}
    \begin{vmatrix}
    1 & 2 & 3 & 4 \\
    x & 1 & 2 & 3 \\
    x & x & 1 & 2 \\
    x & x & x & 1
    \end{vmatrix}
    &\xlongequal[i=2,3,4]{C_i = C_i - iC_1}
    \begin{vmatrix}
    1 & 0    & 0    & 0    \\
    x & 1-2x & 2-3x & 3-4x \\
    x & -x   & 1-3x & 2-4x \\
    x & -x   & -2x  & 1-4x
    \end{vmatrix}
    =
    \begin{vmatrix}
    1-2x & 2-3x & 3-4x \\
    -x   & 1-3x & 2-4x \\
    -x   & -2x  & 1-4x
    \end{vmatrix} \\
    &\xlongequal[R_2 = R_2-R_3]{R_1 = R_1-2R_2}
    \begin{vmatrix}
    1 & 3x & -1+4x \\
    0 & 1-x & 1 \\
    -x & -2x & 1-4x
    \end{vmatrix}
    = -4x^3+6x^2-4x+1
    \end{align*}
\end{solution}

\begin{exercise}[14--15 秋冬]
    计算行列式
    \[
        D = \begin{vmatrix}
            1 & 2 & 3 & \cdots & n - 1 & n \\
            x & 1 & 2 & \cdots & n - 2 & n - 1 \\
            x & x & 1 & \cdots & n - 3 & n - 2 \\
            \vdots & \vdots & \vdots & \ddots & \vdots & \vdots \\
            x & x & x & \cdots & 1 & 2 \\
            x & x & x & \cdots & x & 1
        \end{vmatrix}.
    \]
\end{exercise}

\begin{solution}
    这是习题~\ref{ex:vec:123x} 的一般化:
    \begin{align*}
        D 
        &\xlongequal[i = 1, 2, \ldots, n - 1]{R_i = R_i - R_{i + 1}}
        \begin{vmatrix}
            1 - x & 1 & 1 & \cdots & 1 & 1 \\
            0 & 1 - x & 1 & \cdots & 1 & 1 \\
            0 & 0 & x - 1 & \cdots & 1 & 1 \\
            \vdots & \vdots & \vdots & \ddots & \vdots & \vdots \\
            0 & 0 & 0 & \cdots & 1 - x & 1 \\
            x & x & x & \cdots & x & 1 \\
        \end{vmatrix} \\[3pt]
        &\xlongequal[i = n, n-1, \ldots, 2]{C_i = C_i - C_{i - 1}}
        \begin{vmatrix}
            1 - x & x & 0 & \cdots & 0 & 0 \\
            0 & 1 - x & x & \cdots & 0 & 0 \\
            0 & 0 & 1 - x & \cdots & 0 & 0 \\
            \vdots & \vdots & \vdots & \ddots & \vdots & \vdots \\
            0 & 0 & 0 & \cdots & 1 - x & x \\
            x & 0 & 0 & \cdots & 0 & 1 - x \\
        \end{vmatrix} \\[3pt]
        &\xlongequal{\text{对第一列展开}} (1 - x)^n + (-1)^{n + 1} x^n.
    \end{align*}
\end{solution}

\begin{exercise}
    计算行列式
    \[\begin{vmatrix}
        1 & 1 & 1 & \cdots & 1 \\
        1 & a_1 & 0 & \cdots & 0 \\
        1 & 0 & a_2 & \cdots & 0 \\
        \vdots & \vdots & \vdots & \ddots & \vdots \\
        1 & 0 & 0 & \cdots & a_n \\
    \end{vmatrix},\]
    其中 $a_i \neq 0$, $i= 1, 2, \ldots, n$.
\end{exercise}

\begin{solution}
    凑成上三角行列式:
    \[
    \begin{vmatrix}
    1 & 1 & 1 & \cdots & 1 \\
    1 & a_1 & 0 & \cdots & 0 \\
    1 & 0 & a_2 & \cdots & 0 \\
    \vdots & \vdots & \vdots & \ddots & \vdots \\
    1 & 0 & 0 & \cdots & a_n \\
    \end{vmatrix} \xlongequal{C_1 = C_1 - \sum_{i=1}^{n} (C_{i+1} / a_i)}
    \begin{vmatrix}
    a_0 & 1 & 1 & \cdots & 1 \\
    0 & a_1 & 0 & \cdots & 0 \\
    0 & 0 & a_2 & \cdots & 0 \\
    \vdots & \vdots & \vdots & \ddots & \vdots \\
    0 & 0 & 0 & \cdots & a_n \\
    \end{vmatrix} = \prod_{i=0}^{n} a_i,
    \]
    其中 $a_0 = 1 - \sum_{i=1}^{n} (1/a_i)$.
\end{solution}

\subsection{矩阵的基本运算,可逆矩阵}

\begin{exercise}
    设 $A$, $B$, $A + B$ 均为 $n$ 阶可逆矩阵,证明 $A^{-1} + B^{-1}$ 为可逆矩阵,并写出它的逆。
\end{exercise}

\begin{solution}
    先对 $A^{-1} + B^{-1}$ 左乘 $A$, 右乘 $B$, 得到
    \[
        A(A^{-1} + B^{-1})B = B + A = A + B.
    \]
    两侧同时取逆,由 $(AB)^{-1} = B^{-1} A^{-1}$, 我们有
    \[
        B^{-1} (A^{-1} + B^{-1})^{-1} A^{-1}  = (A + B)^{-1}.
    \]
    由此可知 $A^{-1} + B^{-1}$ 可逆, 且其逆矩阵为 $B (A + B)^{-1} A$.
\end{solution}

\begin{exercise}
	已知 $A = [a_{ij}]_{n \times n}$ 是可逆矩阵. 求其伴随矩阵对应行列式的值 $\abs{A^*}$.
\end{exercise}
\begin{solution}
	因为 $A^{-1} = \abs{A}^{-1} A^*$ (逆矩阵的一种定义), 我们有
    \[
        \abs{A^*} 
        = \abs[1]{\, \abs{A} A^{-1} } 
        = \abs{A}^n \cdot \abs{A^{-1}} 
        = \abs{A}^{n-1}.
    \]
    (注意到 $\abs{A}$ 是一个常数; 当$\lambda$ 是常数时, $\lambda A = [\lambda a_{ij}]_{n \times n}$, $\abs{\lambda A} = \lambda^n \abs{A}$.)
\end{solution}

\begin{exercise}
    设 $3$ 阶实对称矩阵 $A$ 的各行元素之和均为 $3$, 向量 $\vect{\alpha}_1 = (-1\quad 2\quad {-1})^T$, $\vect{\alpha}_2 = (0\quad {-1}\quad 1)^T$ 是线性方程组 $Ax = 0$ 的两个解.
    \begin{enumerate}[(1)]
        \item 求 $A$ 的特征值, 
        \item 求 $A$ 及 $\abs[1]{A-\tfrac 32 E}^6$.
    \end{enumerate}
\end{exercise}

\begin{solution}
    \begin{enumerate}[(1)]
        \item $0$ 是 $A$ 的三重特征根.
        \item 由 $A$ 是三阶实对称矩阵, 易得
        \[
        A = \begin{bmatrix}
        1 & 1 & 1\\ 1 & 1 & 1\\ 1 & 1 & 1
        \end{bmatrix}.
        \]
        记
        \[
        B = A - \tfrac 32 E = \frac 12\begin{bmatrix}
        -1 & 2 & 2 \\ 2 & -1 & 2 \\ 2 & 2 & -1
        \end{bmatrix} = Q\Lambda Q^{-1},
        \]
        其中
        \begin{align*}
        \Lambda &= \frac 12\begin{bmatrix}
        3 & 0 & 0 \\ 0 & -3 & 0 \\ 0 & 0 & -3
        \end{bmatrix} \text{\ 是由 $B$ 的特征值构成的对角矩阵}, \\
        Q &= \begin{bmatrix}
        1 & 1 & 1 \\ 1 & 0 & -1 \\ 1 & -1 & 0
        \end{bmatrix} \text{\ 由 $B$ 的特征根对应的特征向量(行向量)构成}, \\
        Q^{-1} &= \frac 13\begin{bmatrix}
        1 & 1 & 1 \\ 1 & 1 & -2 \\ 1 & -2 & 1
        \end{bmatrix}.
        \end{align*}
        所以
        \begin{align*}
            \abs[2]{A - \tfrac 32 E}^6 
            = \abs{B}^6 
            = \abs[1]{Q \Lambda Q^{-1}}^6 
            = \abs[1]{Q \Lambda^6 Q^{-1}} 
            = \abs[1]{Q} \cdot \abs[1]{\Lambda^6} \cdot \abs[1]{Q^{-1}}.
        \end{align*}
        (后面自己算吧)
    \end{enumerate}
\end{solution}

\subsection{线性方程组的解,解空间}
\begin{exercise}
    设 $\vect{\alpha}_1$, $\vect{\alpha}_2$, \ldots, $\vect{\alpha}_s$ 是某欧氏空间中不含零向量的正交向量组, 非零向量 $\vect{\beta}$ 是它们的线性组合。试求 $\sum_{i = 1}^s \cos^2 \ang{\vect{\beta}, \vect{\alpha}_i}$, 这里 $\ang{\vect{\beta}, \vect{\alpha}_i}$ 表示向量 $\vect{\beta}$ 与向量 $\vect{\alpha}_i$ 的夹角, $i = 1, 2, \ldots, s$.
\end{exercise}
    
\begin{solution}
    记 $\vect{\beta} = \sum_{i = 1}^s k_i \vect{\alpha}_i$, 其中 $\sum_{i = 1}^s k_i^2 > 0$, $\norm{\vect{\alpha}_i} > 0$, 则有
    \[
        \norm{\vect{\beta}}^2 = \sum_{i = 1}^s k_i^2 \norm{\vect{\alpha}_i}^2
        \qquad\text{及}\qquad
        \cos \ang{\vect{\beta}, \vect{\alpha}_i} 
        = \frac{\vect{\beta} \cdot \vect{\alpha}_i}{\norm{\vect{\beta}} \cdot {\norm{\vect{\alpha}_i}}}
        = \frac{k_i \norm{\vect{\alpha}_i}^2}{\norm{\vect{\beta}} \cdot {\norm{\vect{\alpha}_i}}}
        = \frac{k_i \norm{\vect{\alpha}_i}}{\norm{\vect{\beta}}},
    \]
    于是
    \[
        \sum_{i = 1}^s \cos^2 \ang{\vect{\beta}, \vect{\alpha}_i}
        = \sum_{i = 1}^s \frac{k_i^2 \norm{\vect{\alpha}_i}^2}{\norm{\vect{\beta}}^2}
        = \frac{ \norm{\vect{\beta}}^2 }{ \norm{\vect{\beta}}^2 }
        = 1.
    \]
\end{solution}

\begin{exercise}
    定义矩阵序列 $\set{(b_y^k)_{m \times n}}_{k \ge 1}$ 的极限为:$\lim_{k \to \infty} (b_y^k)_{m \times n} = (\lim_{k \to \infty} (b_y^k))_{m \times n}$. 设 $2$ 阶矩阵
    \[
        A = \begin{bmatrix}
            1 - a & a \\
            b & 1 - b
        \end{bmatrix},
    \]
    其中 $0 < a + b < 1$, 求 $\lim_{k \to \infty} A^k$.
\end{exercise}

\begin{solution}
    记
    \[
        B
        = A - E
        = \begin{bmatrix}
            - a & a \\
            b & - b
        \end{bmatrix},
    \]
    易得 $B^2 = -(a + b)B$. 所以
    \[
        A^k = (E + B)^k = E + \del[2]{\sum_{i = 1}^k C_k^i (-1)^{i - 1} (a + b)^{i - 1}}B
        = E - \frac{(1-a-b)^k - 1}{a + b} B,
    \]
    其中
    \begin{align*}
        \sum_{i = 1}^k C_k^i (-1)^{i - 1} (a + b)^{i - 1}
        &= -\frac{1}{a + b} \sum_{i = 1}^k C_k^i (-1)^{i} (a + b)^{i} \\
        &= -\frac{1}{a + b} \sbr[3]{\sum_{i = 0}^k C_k^i (-1)^{i} (a + b)^{i} - 1} \\
        &= -\frac{(1-a-b)^k - 1}{a + b}.
    \end{align*}
    因为 $0 < a + b < 1$, $\lim_{k \to \infty} (1 - a - b)^k = 0$, 所以
    \[
        \lim_{k \to \infty} A^k = E + \frac{1}{a + b} B 
        = \frac{1}{a + b} \begin{bmatrix}
            b & a \\
            b & a
        \end{bmatrix}.
    \]
\end{solution}